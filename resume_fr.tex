%% start of file `template.tex'.
%% Copyright 2006-2010 Xavier Danaux (xdanaux@gmail.com).
%
% This work may be distributed and/or modified under the
% conditions of the LaTeX Project Public License version 1.3c,
% available at http://www.latex-project.org/lppl/.


\documentclass[11pt,a4paper]{moderncv}
\usepackage{bibentry}
\usepackage[T1]{fontenc}
\usepackage{color}
%\usepackage[]{babel}
% moderncv themes
\moderncvtheme[green]{classic}
\usepackage[utf8]{inputenc}
\usepackage{wasysym}


% adjust the page margins
\usepackage[scale=0.92]{geometry}

\firstname{Noé}
\familyname{Gaumont}
\title{Postdoctorat à l'Institut des Systèmes Complexes} 
%\extrainfo{Permis B}

\address{313 rue des Pyrénées boite 43}{75020 Paris} 
\mobile{+33 6 77 79 86 28}
\email{noe@ngaumont.fr}
%\photo[72pt]{"image"}


\nopagenumbers{} 

\begin{document}
\maketitle
\vspace{-1.2cm}

\section{Expériences professionnelles}
\cventry{Oct. 2016 - maintenant} {Postdoctorat} {Centre d’Analyse et de Mathématique Sociales (CAMS)}{}{résident à l'Institut des Sytèmes Complexes (ISC), Paris}
{
	Étude de la structure de réseaux tel que Twitter pour le porjet PolitoscopeStudy. Le but est de détecter les communautés de militants politiques sur Twiter mais surtout de suivre leurs évolutions au cours du temps. Cela permet aux journalistes et aux citoyens de mieux comprendre l'organisation de Twitter sur les sujets politiques.
}
\vspace*{0.2cm}
\cventry{Oct. 2013 - Oct. 2016} {Doctorat} {Université Pierre et Marie Curie}{}{dans l'équipe ComplexNetwork, LIP6, Paris}
{
 Étude sur la détection de communautés dans les flots de liens. Les flots de liens sont un outil pour étudier les réseaux temporels.
 Un flot de liens est défini par une séquence d'interactions temporelles, comme les emails. Dans ce contexte, une communauté est un sous-flot de lien définit par des liens et non par des n\oe uds.
}
\vspace*{0.2cm}
\cventry{Fév. 2013 - Juillet 2013}{Projet de fin d'étude}{Thales Air System dans  l'Innovation Lab}{Rungis}{}{Étude et optimisation de la prédictibilité des points caractéristiques d’un vol\newline \textit{Key concepts:} machine learning, data extrapolation. \textit{Languages:} C++, R.}%
\vspace*{0.2cm}
\cventry{Sept. 2011 - Fév. 2012}{Stage assitant-ingénieur}{Commissariat à l'énergie atomique (CEA)}{Brétigny-sur-Orge}{}{Conception et développement d'un algorithme générant un maillage quadrangulaire sous contraintes géométriques et d'un champ de direction.
\textit{Key concepts}: paving mesh generation, finite elements. \textit{Language}: C++.}

%\cventry{Jan 2009 - Feb 2009}{Worker internship}{Sealed Air - quality department}{\'Epernon}{}{Product control within the scope of quality check and communication with clients.\newline}%

%\cvline{April 2010 - Aug 2010}{Exchange program in Germany at the Technische Universität Hamburg-Harburg (TUHH).\newline Introduction to the finite element method and to the structural aspect of planes.}

\section{Éducation}
\cventry{Sept. 2008 - Juillet 2013}{École d'ingénieur}{Université de Technologie de Compiègne}{}{en Informatique, Compiègne}
{%IT project examples carried out during my university training:
%\begin{itemize}\renewcommand{\labelitemi}{$\; \; \; \; \;   \circ$}	
%	\item Development in C++ of meta-heuristics to solve the 2D bin packing problem under guillotine constraints.
%	\item Development of the simplex algorithm in scilab.
%	\item Decentralized chess game developed in Java with a group of \textit{23} people.
	%\item Development of a Gephi plugin to solve the multi-source vehicle routing problem with heuristics.
%	\item Conception of a Tower Defense game in C++ and Qt.\newline
%\end{itemize}
}
\cvline{Juin 2008}{\textbf{Baccalauréat S-SVT}, spécialité mathématique, mention très bien au lycée \textsl{Fulbert}, Chartres.}

\section{Compétence}

\cvline{Mathématique}{Graph theory, complex systems, mathematical optimization, meta-heuristics, constrained programming, basics in cryptography.}

\cvline{Codage}{ \vspace*{-0.25cm}
\begin{description}
\item[Langages:]{Rust, C++, Python, Scala, Spark,  PostgreSQL.}
\item[Otuils:]{Git/svn, Gephi/Tulip, Scilab.}
\item[Web:]{HTML, JavaScript, CSS, PHP.}
\end{description}\vspace*{-1cm}
}
\vspace*{-0.55cm}
%\tiny{\CIRCLE \LEFTcircle}
%\bibliographystyle{plain}
%\bibliography{library.bib}
%\bibentry{Gaumont2016}
%\bibentry{Gaumont2015}
%\bibentry{Gaumont2014}
\section{Publications}
\subsection{\hspace*{0.2cm}En cours de soumission}
\cvline{[1]}{ Noe Gaumont, Maziyar Panahi and David Chavalarias. Methods for the reconstruction of the socio-semantic dynamics of political activist Twitter networks: Application to the 2017 French Presidential elections . Soumis à  \emph{PlosOne}: \url{https://hal.archives-ouvertes.fr/hal-01575456v2}}

\subsection{\hspace*{0.2cm}Journal international}
\cvline{[2]}{Noé Gaumont, Clémence Magnien and Matthieu Latapy. Finding remarkably dense sequences of contacts in link streams. \emph{Social Network Analysis and Mining}, 6(1), 87: \url{https://hal.archives-ouvertes.fr/hal-01390043}}

\subsection{\hspace*{0.2cm}Conférence international}
\cvline{[3]}{Noé Gaumont, Tiphaine Viard, Raphaél Fournier-S'niehotta, Qinna Wang and Matthieu Latapy. Analysis of the temporal and structural features of threads in a mailing-list. In \emph{Complex Networks VII}, Dijon, France. 2016. \emph{Acceptation rate: 23\%}: \url{https://hal.archives-ouvertes.fr/hal-01345821}}
\cvline{[4]}{Noé Gaumont, François Queyroi, Clémence Magnien and Matthieu Latapy. Expected Nodes: a quality function for the detection of link communities. In \emph{Complex Networks VI}, New-York, USA. 2015. Long version of [5]. \emph{Acceptation rate: 20\%}:\url{http://hal.upmc.fr/hal-01196796}}
\subsection{\hspace*{0.2cm}Conférence nationale}
\cvline{[5]}{Noé Gaumont and François Queyroi. Partitionnement des liens d'un graphe : Critéres et Mesures. In \emph{Algotel - 16èmes Rencontres francophones sur les Aspects Algorithmiues des Télécommunication}, Ile de ré, France. 2014. \emph{Acceptation rate: 55\%}: \url{https://hal.archives-ouvertes.fr/hal-00986216}}
\cvline{[6]}{Noé Gaumont. Trouver des séquences de contacts pertinentes dans un flot de liens. In \emph{Algotel - 18èmes Rencontres francophones sur les Aspects Algorithmiues des Télécommunication}, Bayonne, France. 2016. Short version of [2]. \emph{Acceptation rate: 60\%}: \url{https://hal.archives-ouvertes.fr/hal-01305118}}


\section{Présentations}

\subsection{\hspace*{0.2cm}Audience international}
%\cvline{}{Masterclass with Crowcroft january 2016.}
%\cvline{}{Rescom january 2016}

\cvline{[7]}{Noé Gaumont, Maziyar Panahi and David Chavalarias. \emph{Evolution of communities on twitter during the 2017 French presidential election} in Conference Complex Systems (CCS) . 2017: \url{http://easychair.org/smart-program/CCS'17/2017-09-18.html\#talk:47444}}


\cvline{[8]}{Noé Gaumont, Clémence Magnien and Matthieu Latapy. \emph{Bringing density to link streams reveals meaningful groups in contact traces} in workshop e-Young Researchers Network in Complex Systems. 2015: \url{http://cs-dc-15.org/e-tracks/global/\#yr}}


\cvline{[9]}{Tiphaine Viard and Noé Gaumont.\emph{LinkStreamViz: a drawing tool for link stream}. In \emph{Workshop Dynamics On and Of networks}. 2016: \url{https://project.inria.fr/netspringlyon/3-workshops-on-network-sciences/workshop-on-processes-on-and-of-networks/}}

\subsection{\hspace*{0.2cm}Audience nationale}

\cvline{[10]}{Noé Gaumont.\emph{Utilisation de flots de liens pour étudier les interactions temporelles}, 24e journées thématique de Rochebrune 2017}
\cvline{[11]}{Noé Gaumont. \emph{Tools to study link streams}, in workshop  Outils d'analyse de la dynamique temporelle dans les réseaux in Toulouse, France. 2016: \url{http://xsys.fr/wp-content/uploads/2016/09/journe\%CC\%81e-du-14-decembre.pdf}}

%\cvline{}{Presentation RNSC?}

\section{Responsabilités scientifiques}
\cvline{}{Membre du commité d'organisation de MARAMI 2014 et ASONAM 2015.}
\cvline{}{Relecteur pour: SITIS 2015, WWW 2015, Algotel 2016, ICDE 2016 et Journal of Complex Networks.}
\section{Enseignements et vulgarisations}
\cvitem{L1}{programmation impérative et éléments d'algorithme en C (40h TD + 60h TP)}
\cvline{L1}{Éléments de programmation (Python) (20h TP)}
\cvline{L2}{Programmation et structures de données en C (20h TP)}
\cvline{L2}{Introduction aux bases de données relationnelles (20h TP)}

\cvline{}{Animations volontaires:
	\begin{itemize}
	\item[$\circ$] \emph{Apprentissage de HTML et CSS avec des cubes en papier et Thimble}, pour  enfants et adultes dans les bibliothèques
	\item[$\circ$] \emph{Gestion de la vie Privé sur le web}, pour enfants et adultes dans les bibliothèques.
	\item[$\circ$] Présentation du politoscope  lors de l'inauguration de l'exposition Tera Data à la Cité des Sciences. \url{http://www.cite-sciences.fr/fr/au-programme/expos-temporaires/terra-data/}
	\item[$\circ$] Présentation de projets de l'ISC dont le politoscope à Innovatives SHS, un salon de valoriation des sciences humaines et sociales à Marseille \url{http://innovatives.cnrs.fr/innovatives-shs-2017/exposition/article/expertise}
	\item[$\circ$] Tiphaine Viard and Noé Gaumont. \emph{De la découverte au partage dans la recherche en informatique}, au CoFestival, un événement inclusif les savoir-faire scientifiques et technologiques. \url{http://web.archive.org/web/20160111020002/http://cofestival.org/\#programme}
	\end{itemize}}
\vspace*{-0.6cm}
%\section{Training}
%\cvline{}{School on structure and dynamics of complex networks (2 weeks)}
%\cvline{}{Rescom 2014 : Network Science (1 week)}


\section{Langues \hspace{6.4cm} Intérêts personnels \hfill}
%\cvline{French}{Mother tongue}{}
\begin{minipage}{0.5\textwidth}
\cvline{Anglais}{European level C1.\newline $\circ$ \textit{Score au TOEIC en 2012: 960/990}.}
\cvline{Allemand}{European level B2. Connaissance basique.}
\end{minipage}
\hspace*{0.06\textwidth}
\begin{minipage}{0.4\textwidth}
\vspace*{-0.17cm}
Open-source software (Mozilla), vie privé sur internet, sport (escalade, badminton).
\end{minipage}



\end{document}


%% end of file `template_en.tex'.